\chapter{Introduction}
%Aktuálnost tématu
The year is 2020. I can say with confidence, that the last decade belonged to AI. Modern hardware finally allowed for solving tasks deemed too computationally demanding before. Machine learning algorithms are already present in our daily activities, even if we do not know about them. They are responsible for solving numerous tasks: predictions, recommendations, object detection and recognition and many more.  

%Čím je téma prospěšné pro společnost a proč se problematikou zabýváme
I have always wanted to connect my field (computer science) with natural sciences. Medical industry offers various applications of machine learning, one of which is image analysis. Medical images have become an integral part of diagnosis and treatment of patients. It comes just natural to try to automize the process of their analysis and diagnosis. 

%Stanovenícíle(ů)práce
The focus of my thesis is detection of organs in CT images using modern approaches, such as neural nets. This covers researching the medical imaging field, state-of-the-art methods for image analysis and also implementing my own prototype for organs detection with the data set of my choice. 

%Představení struktury celé práce – obsah jednotlivých kapitol
In the first chapter of my thesis I will talk about how the field of medical imaging developed throughout the history, starting with X-Rays. I will focus on classical methods used for medical image processing, which proved to be efficient, but are being replaced by machine learning algorithms. The following chapter introduces CT as a technology and some important characteristics of the CT scans. The third chapter is a review of convolutional neural networks and their variations proposed for different scenarios and uses in image processing. And last, but not least, I will describe the work I have done implementing a neural networks model for detection of organs in CT images. My task is segmentation of a spine and specific vertebrae. 