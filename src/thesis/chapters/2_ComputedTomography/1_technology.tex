\section{Technology}
Technology behind CT scanning is based on X-Ray. X-Ray uses a tube in a fixed position which sends x-rays. In comparison to CT, where the tube moves around the patient in a so called translate-rotate motion. 

CT consists of an x-ray tube, which emits x-ray beams and an x-ray detector, which measures the x-ray transmission. The detector is located on the other side of the patient's body, just across from the tube. 

Patient's body can be divided into slices. You can imagine them as slices of bread. Thickness of these slices is determined by the thickness of the beam used for scanning. CT scanner performs the following steps on every slice. 

Both the tube and the detector move along the slice in one direction, performing a certain number of measurements. This phase is called translation. All the measurements collected during one translation phase are called a view. 

The rotation phase comes after every translation phase. The tube (together with the detector) rotates by 1°, so a view from a different angle can be obtained in the next translation phase. Hence, the name translate-rotate motion. After views for all the angles have been collected, the whole translate-rotate procedure is repeated on the next slice.

This is a basic principle of CT initially proposed by Haunsfield. His prototype called Mark I was able to collect 180 views over 180° (one view per 1° increment) and 160 measurements per view, which totalled in 28,800 measurements per whole scanning procedure. \cite{goldman2007}







